\documentclass[12pt, a4paper]{article}
\usepackage[utf8]{inputenc}
\usepackage[T1]{fontenc}
\usepackage[slovene]{babel}
\usepackage{lmodern}
\usepackage{amsmath}
\usepackage{eurosym}
\usepackage{amsfonts}
\usepackage{hyperref}
\usepackage{tikz}


\usepackage{enumerate}
\setlength{\parindent}{0mm}

\newcommand{\novukaz}[2]{\underline {#1} \textit{#2}}

\newcounter{stevec}

\newenvironment{novookolje}[2]{\stepcounter{stevec} #1 #2 \thestevec}{}


\begin{document}
\begin{titlepage}
\begin{center}

\large
Univerza v Ljubljani\\
\normalsize
Fakulteta za matematiko in fiziko\\

\vspace{3 cm} 

\large
Eva Deželak, Ines Šilc in Žan Jarc\\

\vspace{0.5cm}
\LARGE
\textbf{PRORAČUN RS}

\vspace{0.5 cm}
\normalsize


\vspace{1.5cm}
\normalsize
Mentor: doc. dr. Matjaž Črnigoj

\vspace{3cm}


\vfill

\large Ljubljana, 2019

\end{center}
\end{titlepage}



\newpage
\section*{Povzetek}


\newpage

\tableofcontents
\vspace{20mm}

\newpage
\section[Uvod]{Uvod}
V seminarski nalogi bomo predstavili proračun Republike Slovenije. Najprej se bomo osredotočili na to, kaj sploh je proračun, kako je sestavljen in kako poteka njegova priprava. Nato bomo konkretneje preverili proračun za leto 2019 ter se osredotočili na razdelke, kjer bo viden večji premik v primerjavi s prejšnjimi leti. V naš okvir obravnave bomo namreč vzeli proračune RS med leti 2015 in 2019. Hkrati pa se nam bo pojavljajo še vprašanje demografije in kako to vpliva na sam proračun RS. V naslednjem koraku bomo poskusili narediti še grobo napoved za naprej. \\
\\
V zadnjem korakupa  bomo primerjali proračun Republike Slovenije s proračunom ostalih držav, kjer bomo poskušali najti tiste, ki so dobro primerljive tudi na razvojnem področju Slovenije. 





\end{document}