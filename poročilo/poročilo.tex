\documentclass[12pt, a4paper]{article}
\usepackage[utf8]{inputenc}
\usepackage[T1]{fontenc}
\usepackage[slovene]{babel}
\usepackage{lmodern}
\usepackage{amsmath}
\usepackage{eurosym}
\usepackage{amsfonts}
\usepackage{hyperref}
\usepackage{tikz}


\usepackage{enumerate}
\setlength{\parindent}{0mm}

\newcommand{\novukaz}[2]{\underline {#1} \textit{#2}}

\newcounter{stevec}

\newenvironment{novookolje}[2]{\stepcounter{stevec} #1 #2 \thestevec}{}


\begin{document}
\begin{titlepage}
\begin{center}

\large
Univerza v Ljubljani\\
\normalsize
Fakulteta za matematiko in fiziko\\

\vspace{3 cm} 

\large
Eva Deželak, Žan Jarc in Ines Šilc\\

\vspace{0.5cm}
\LARGE
\textbf{PRORAČUN RS}

\vspace{0.5 cm}
\normalsize


\vspace{1.5cm}
\normalsize
Mentor: doc. dr. Matjaž Črnigoj

\vspace{3cm}


\vfill

\large Ljubljana, 2019

\end{center}
\end{titlepage}



\newpage
\section*{Povzetek}


\newpage

\tableofcontents
\vspace{20mm}

\newpage
\section[Uvod]{Uvod}
V seminarski nalogi bomo predstavili proračun Republike Slovenije. Najprej se bomo osredotočili na to, kaj sploh je proračun, kako je sestavljen in kako poteka njegova priprava. Nato bomo konkretneje preverili proračun za leto 2019 ter se osredotočili na razdelke, kjer bo viden večji premik v primerjavi s prejšnjimi leti. V naš okvir obravnave bomo namreč vzeli proračune RS med leti 2015 in 2019. Hkrati pa se nam bo pojavljajo še vprašanje demografije in kako to vpliva na sam proračun RS. V naslednjem koraku bomo poskusili narediti še grobo napoved za naprej. \\
\\
V zadnjem koraku pa  bomo primerjali proračun Republike Slovenije s proračunom ostalih držav, kjer bomo poskušali najti tiste, ki so dobro primerljive tudi na razvojnem področju Slovenije. 

\section[Splošno o proračunu RS]{Splošno o proračunu RS}
V preteklosti je izraz "budžet"  v Slovenščino prodrl iz Angleškega jezika, kjer je prvotno predstavljal torbo, v kateri je imel kralj spravljen denar za javne izdatke. \\
Danes se izraz "proračun"  uporablja le za letni načrt prihodkov in odhodkov družbeno - političnih skupnosti (npr. državni, občinski proračun).
 \\
\\
Proračun Republike Slovenije je akt države, s katerim so predvideni vsi prihodki in drugi prejemki ter odhodki in drugi izdatki države za eno leto. Proračun sprejme Državni zbor po posebnem, predpisanem postopku.
\\
\\
Državni proračun je pomemben instrument, ki ga ima vlada na voljo pri izvajanju večletne makroekonomske politike, katere cilj je zagotavljanje stabilnih javnih financ in pospeševanje gospodarskega ter družbenega razvoja. Temeljne naloge pri upravljanju proračuna so uresničitev proračuna v okvirih in za namene, kot je bil sprejet, njegovo pravočasno in fleksibilno prilagajanje spremenjenim fiskalnim okoliščinam in uresničevanje v proračunu zastavljenih družbenih in gospodarskih ciljev. \ref{Splošno o proračunu}


\section[Viri]{Viri}
\begin{enumerate}
\item
\label{Splošno o proračunu}
Ministrstvo za finance. \emph{Splošno o proračunu.} Dostopno na: \url{http://www.mf.gov.si/si/delovna_podrocja/proracun/splosno_o_proracunu/},  (ogled 9.~4.~2019)
\end{enumerate}


\end{document}