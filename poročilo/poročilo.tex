\documentclass[12pt, a4paper]{article}
\usepackage[utf8]{inputenc}
\usepackage[T1]{fontenc}
\usepackage[slovene]{babel}
\usepackage{lmodern}
\usepackage{amsmath}
\usepackage{eurosym}
\usepackage{amsfonts}
\usepackage{hyperref}
\usepackage{tikz}


\usepackage{enumerate}
\setlength{\parindent}{0mm}

\newcommand{\novukaz}[2]{\underline {#1} \textit{#2}}

\newcounter{stevec}

\newenvironment{novookolje}[2]{\stepcounter{stevec} #1 #2 \thestevec}{}


\begin{document}
\begin{titlepage}
\begin{center}

\large
Univerza v Ljubljani\\
\normalsize
Fakulteta za matematiko in fiziko\\

\vspace{3 cm} 

\large
Eva Deželak, Žan Jarc in Ines Šilc\\

\vspace{0.5cm}
\LARGE
\textbf{PRORAČUN RS}

\vspace{0.5 cm}
\normalsize


\vspace{1.5cm}
\normalsize
Mentor: doc. dr. Matjaž Črnigoj

\vspace{3cm}


\vfill

\large Ljubljana, 2019

\end{center}
\end{titlepage}



\newpage
\section*{Povzetek}


\newpage

\tableofcontents
\vspace{20mm}

\newpage
\section[Uvod]{Uvod}
V seminarski nalogi bomo predstavili proračun Republike Slovenije. Najprej se bomo osredotočili na to, kaj sploh je proračun, kako je sestavljen in kako poteka njegova priprava. Nato bomo konkretneje preverili proračun za leto 2019 ter se osredotočili na razdelke, kjer bo viden večji premik v primerjavi s prejšnjimi leti. V naš okvir obravnave bomo namreč vzeli proračune RS med leti 2015 in 2019. Hkrati pa se nam bo pojavljajo še vprašanje demografije in kako to vpliva na sam proračun RS. V naslednjem koraku bomo poskusili narediti še grobo napoved za naprej. \\
\\
V zadnjem koraku pa  bomo primerjali proračun Republike Slovenije s proračunom ostalih držav, kjer bomo poskušali najti tiste, ki so dobro primerljive tudi na razvojnem področju Slovenije. 
\\
\\
\section[Splošno o proračunu RS]{Splošno o proračunu RS}
V preteklosti je izraz "budžet" v Slovenščino prodrl iz Angleškega jezika, kjer je prvotno predstavljal torbo, v kateri je imel kralj spravljen denar za javne izdatke. \\
Danes se izraz "proračun" uporablja le za letni načrt prihodkov in odhodkov družbeno - političnih skupnosti (npr. državni, občinski proračun).
 \\
\\
Proračun Republike Slovenije je akt države, s katerim so predvideni vsi prihodki in drugi prejemki ter odhodki in drugi izdatki države za eno leto. Proračun sprejme Državni zbor po posebnem, predpisanem postopku.
\\
\\
Državni proračun je pomemben instrument, ki ga ima vlada na voljo pri izvajanju večletne makroekonomske politike, katere cilj je zagotavljanje stabilnih javnih financ in pospeševanje gospodarskega ter družbenega razvoja. Temeljne naloge pri upravljanju proračuna so uresničitev proračuna v okvirih in za namene, kot je bil sprejet, njegovo pravočasno in fleksibilno prilagajanje spremenjenim fiskalnim okoliščinam in uresničevanje v proračunu zastavljenih družbenih in gospodarskih ciljev. [\ref{Splošno o proračunu}]

\subsection[Sestavni deli proračuna]{Sestavni deli proračuna}
\subsubsection[Splošno del proračuna]{Splošni del proračuna}
Splošni del proračuna vključuje bilanco prihodkov in odhodkov, račun finančnih terjatev in naložb in račun financiranja.
\begin{itemize}
\item V \textbf{bilanci prihodkov in odhodkov} se izkazujejo prihodki, ki obsegajo tekoče prihodke (davčne in nedavčne prihodke), kapitalske prihodke, prejete donacije ter transferne prihodke iz drugih blagajn javnega financiranja. Na strani odhodkov pa se v izkazujejo odhodki, ki zajemajo tekoče odhodke, tekoče transfere, investicijske odhodke ter investicijske transfere.

\item \textbf{Račun finančnih terjatev} zajema na strani izdatkov tiste tokove izdatkov, ki za državo nimajo značaja odhodkov (to je nepovratno danih sredstev), pač pa imajo bodisi značaj danih posojil finančnih naložb oziroma kapitalskih vlog države v javna in zasebna podjetja, banke oziroma druge finančne institucije. Plačila imajo za rezultat nastanek finančne terjatve države (ali občine) do prejemnika teh sredstev ali pa vzpostavitev oziroma povečanje kapitalskega deleža države v lastniški strukturi prejemnikov teh sredstev. V okviru te skupine izdatkov se izkazujejo tudi plačila zapadlih jamstev države finančnim institucijam ali drugim upnikom, s čimer nastane terjatev države (regresna pravica) do glavnega dolžnika (osebe, za katero je država jamčila). Na strani prejemkov pa so v tem računu izkazani tokovi prejemkov, ki nimajo značaja prihodkov, pač pa so to sredstva iz naslova prejetih vračil posojenih sredstev države oziroma prejetih sredstev iz naslova prodaje kapitalskih deležev države v podjetjih, bankah in drugih finančnih institucijah.

\item V \textbf{računu financiranja} se izkazujejo tokovi zadolževanja in odplačil dolgov, povezanih s servisiranjem dolga države, oziroma s financiranjem proračunskega deficita, to je salda bilance prihodkov in odhodkov ter računa finančnih terjatev in naložb. Zadolževanje je razčlenjeno na najemanje domačih in tujih kreditov ter na sredstva, pridobljena z izdajo državnih vrednostnih papirjev na domačem in tujem trgu. Odplačila dolga pa so razčlenjena na odplačila domačih in tujih kreditov ter odplačila glavnice zapadlih državnih vrednostnih papirjev. V računu financiranja se kot saldo izkazujejo tudi spremembe stanja denarnih sredstev na računih med proračunskim letom.
\end{itemize}

\subsubsection[Posebni del proračuna]{Posebni del proračuna}
Posebni del proračuna pomeni vsebino porabe javnofinančnih sredstev v finančnih načrtih posameznih proračunskih uporabnikov oziroma skupin proračunskih uporabnikov in vključuje odhodke in druge izdatke delovanja predstavljene po politikah, glavnih programih in podprogramih. 

\subsubsection[Načrt razvojnih programov]{Načrt razvojnih programov}
V načrtu razvojnih programov se odhodki načrtujejo po strukturi programske klasifikacije, posameznih ukrepih in projektih ter virih financiranja po posameznih letih za celovito izvedbo projektov in ukrepov. Načrt razvojnih programov se izdela za obdobje celotnega trajanja v načrt vključenih ukrepov in projektov. V proračunu se prikaže po ukrepih, skupinah projektov in projektih ter virih sredstev za njihovo izvedbo. [\ref{Sestavni deli}]

\newpage

\section[Viri]{Viri}
\begin{enumerate}
\item
\label{Splošno o proračunu}
Ministrstvo za finance. \emph{Splošno o proračunu.} Dostopno na: \url{http://www.mf.gov.si/si/delovna_podrocja/proracun/splosno_o_proracunu/},  (ogled 9.~4.~2019).

\item
\label{Sestavni deli}
Ministrstvo za finance. \emph{Sestavni deli proračuna.} Dostopno na: \url{http://www.mf.gov.si/si/delovna_podrocja/proracun/splosno_o_proracunu/sestavni_deli_proracuna/}, (ogled 9.~4.~2019).
\end{enumerate}


\end{document}